% Options for packages loaded elsewhere
\PassOptionsToPackage{unicode}{hyperref}
\PassOptionsToPackage{hyphens}{url}
%
\documentclass[
]{article}
\usepackage{lmodern}
\usepackage{amssymb,amsmath}
\usepackage{ifxetex,ifluatex}
\ifnum 0\ifxetex 1\fi\ifluatex 1\fi=0 % if pdftex
  \usepackage[T1]{fontenc}
  \usepackage[utf8]{inputenc}
  \usepackage{textcomp} % provide euro and other symbols
\else % if luatex or xetex
  \usepackage{unicode-math}
  \defaultfontfeatures{Scale=MatchLowercase}
  \defaultfontfeatures[\rmfamily]{Ligatures=TeX,Scale=1}
\fi
% Use upquote if available, for straight quotes in verbatim environments
\IfFileExists{upquote.sty}{\usepackage{upquote}}{}
\IfFileExists{microtype.sty}{% use microtype if available
  \usepackage[]{microtype}
  \UseMicrotypeSet[protrusion]{basicmath} % disable protrusion for tt fonts
}{}
\makeatletter
\@ifundefined{KOMAClassName}{% if non-KOMA class
  \IfFileExists{parskip.sty}{%
    \usepackage{parskip}
  }{% else
    \setlength{\parindent}{0pt}
    \setlength{\parskip}{6pt plus 2pt minus 1pt}}
}{% if KOMA class
  \KOMAoptions{parskip=half}}
\makeatother
\usepackage{xcolor}
\IfFileExists{xurl.sty}{\usepackage{xurl}}{} % add URL line breaks if available
\IfFileExists{bookmark.sty}{\usepackage{bookmark}}{\usepackage{hyperref}}
\hypersetup{
  pdftitle={Reproducible Research - Course Project 2},
  pdfauthor={Chamodhi Wickramasinghe},
  hidelinks,
  pdfcreator={LaTeX via pandoc}}
\urlstyle{same} % disable monospaced font for URLs
\usepackage[margin=1in]{geometry}
\usepackage{color}
\usepackage{fancyvrb}
\newcommand{\VerbBar}{|}
\newcommand{\VERB}{\Verb[commandchars=\\\{\}]}
\DefineVerbatimEnvironment{Highlighting}{Verbatim}{commandchars=\\\{\}}
% Add ',fontsize=\small' for more characters per line
\usepackage{framed}
\definecolor{shadecolor}{RGB}{248,248,248}
\newenvironment{Shaded}{\begin{snugshade}}{\end{snugshade}}
\newcommand{\AlertTok}[1]{\textcolor[rgb]{0.94,0.16,0.16}{#1}}
\newcommand{\AnnotationTok}[1]{\textcolor[rgb]{0.56,0.35,0.01}{\textbf{\textit{#1}}}}
\newcommand{\AttributeTok}[1]{\textcolor[rgb]{0.77,0.63,0.00}{#1}}
\newcommand{\BaseNTok}[1]{\textcolor[rgb]{0.00,0.00,0.81}{#1}}
\newcommand{\BuiltInTok}[1]{#1}
\newcommand{\CharTok}[1]{\textcolor[rgb]{0.31,0.60,0.02}{#1}}
\newcommand{\CommentTok}[1]{\textcolor[rgb]{0.56,0.35,0.01}{\textit{#1}}}
\newcommand{\CommentVarTok}[1]{\textcolor[rgb]{0.56,0.35,0.01}{\textbf{\textit{#1}}}}
\newcommand{\ConstantTok}[1]{\textcolor[rgb]{0.00,0.00,0.00}{#1}}
\newcommand{\ControlFlowTok}[1]{\textcolor[rgb]{0.13,0.29,0.53}{\textbf{#1}}}
\newcommand{\DataTypeTok}[1]{\textcolor[rgb]{0.13,0.29,0.53}{#1}}
\newcommand{\DecValTok}[1]{\textcolor[rgb]{0.00,0.00,0.81}{#1}}
\newcommand{\DocumentationTok}[1]{\textcolor[rgb]{0.56,0.35,0.01}{\textbf{\textit{#1}}}}
\newcommand{\ErrorTok}[1]{\textcolor[rgb]{0.64,0.00,0.00}{\textbf{#1}}}
\newcommand{\ExtensionTok}[1]{#1}
\newcommand{\FloatTok}[1]{\textcolor[rgb]{0.00,0.00,0.81}{#1}}
\newcommand{\FunctionTok}[1]{\textcolor[rgb]{0.00,0.00,0.00}{#1}}
\newcommand{\ImportTok}[1]{#1}
\newcommand{\InformationTok}[1]{\textcolor[rgb]{0.56,0.35,0.01}{\textbf{\textit{#1}}}}
\newcommand{\KeywordTok}[1]{\textcolor[rgb]{0.13,0.29,0.53}{\textbf{#1}}}
\newcommand{\NormalTok}[1]{#1}
\newcommand{\OperatorTok}[1]{\textcolor[rgb]{0.81,0.36,0.00}{\textbf{#1}}}
\newcommand{\OtherTok}[1]{\textcolor[rgb]{0.56,0.35,0.01}{#1}}
\newcommand{\PreprocessorTok}[1]{\textcolor[rgb]{0.56,0.35,0.01}{\textit{#1}}}
\newcommand{\RegionMarkerTok}[1]{#1}
\newcommand{\SpecialCharTok}[1]{\textcolor[rgb]{0.00,0.00,0.00}{#1}}
\newcommand{\SpecialStringTok}[1]{\textcolor[rgb]{0.31,0.60,0.02}{#1}}
\newcommand{\StringTok}[1]{\textcolor[rgb]{0.31,0.60,0.02}{#1}}
\newcommand{\VariableTok}[1]{\textcolor[rgb]{0.00,0.00,0.00}{#1}}
\newcommand{\VerbatimStringTok}[1]{\textcolor[rgb]{0.31,0.60,0.02}{#1}}
\newcommand{\WarningTok}[1]{\textcolor[rgb]{0.56,0.35,0.01}{\textbf{\textit{#1}}}}
\usepackage{graphicx,grffile}
\makeatletter
\def\maxwidth{\ifdim\Gin@nat@width>\linewidth\linewidth\else\Gin@nat@width\fi}
\def\maxheight{\ifdim\Gin@nat@height>\textheight\textheight\else\Gin@nat@height\fi}
\makeatother
% Scale images if necessary, so that they will not overflow the page
% margins by default, and it is still possible to overwrite the defaults
% using explicit options in \includegraphics[width, height, ...]{}
\setkeys{Gin}{width=\maxwidth,height=\maxheight,keepaspectratio}
% Set default figure placement to htbp
\makeatletter
\def\fps@figure{htbp}
\makeatother
\setlength{\emergencystretch}{3em} % prevent overfull lines
\providecommand{\tightlist}{%
  \setlength{\itemsep}{0pt}\setlength{\parskip}{0pt}}
\setcounter{secnumdepth}{-\maxdimen} % remove section numbering

\title{Reproducible Research - Course Project 2}
\author{Chamodhi Wickramasinghe}
\date{10/21/2020}

\begin{document}
\maketitle

\hypertarget{exploring-the-u.s.-national-oceanic-and-atmospheric-administrations-noaa-storm-database---health-and-economic-impacts}{%
\subsubsection{Exploring the U.S. National Oceanic and Atmospheric
Administration's (NOAA) storm database - Health and Economic
Impacts}\label{exploring-the-u.s.-national-oceanic-and-atmospheric-administrations-noaa-storm-database---health-and-economic-impacts}}

\hypertarget{synopsis}{%
\subsection{Synopsis}\label{synopsis}}

This is a second course project for Reproducible Research course which
is part of the Coursera's Data Science Specialization.

Storms and other severe weather events can cause both public health and
economic problems for communities and municipalities. Many severe events
can result in fatalities, injuries, and property damage, and preventing
such outcomes to the extent possible is a key concern.

This project involves exploring the U.S. National Oceanic and
Atmospheric Administration's (NOAA) storm database. This database tracks
characteristics of major storms and weather events in the United States,
including when and where they occur, as well as estimates of any
fatalities, injuries, and property damage.

The analysis of the data shows that tornadoes, by far, have the greatest
health impact as measured by the number of injuries and fatalities The
analysis also shows that floods cause the greatest economic impact as
measured by property damage and crop damage.

\hypertarget{data-processing}{%
\subsection{Data Processing}\label{data-processing}}

\hypertarget{load-libraries-and-prepare-the-r-environment}{%
\subsection{Load Libraries and prepare the R
environment}\label{load-libraries-and-prepare-the-r-environment}}

\hypertarget{data}{%
\subsection{Data}\label{data}}

The data for this assignment come in the form of a comma-separated-value
file compressed via the bzip2 algorithm to reduce its size. You can
download the file from the course web site:

storm data{[}47Mb{]}

There is also some documentation of the database available. Here you
will find how some of the variables are constructed/defined.

National Weather Service Storm Data Documentation

National Climatic Data Center Storm Events FAQ

The events in the database start in the year 1950 and end in November
2011. In the earlier years of the database there are generally fewer
events recorded, most likely due to a lack of good records. More recent
years should be considered more complete.

\hypertarget{assignment}{%
\subsection{Assignment}\label{assignment}}

The basic goal of this assignment is to explore the NOAA Storm Database
and answer the following basic questions about severe weather events.

Across the United States, which types of events (as indicated in the
EVTYPE variable) are most harmful with respect to population health?
Across the United States, which types of events have the greatest
economic consequences?

\hypertarget{loading-the-data}{%
\subsection{Loading the data}\label{loading-the-data}}

The data was downloaded from the link above and saved on local computer
(in setwd command one can replace loacal file path with path of folder
where the data was downloaded). Then it was loaded on the R using the
read.csv command. If object strom.data is already loaded, use that
cached object insted of loading it each time the Rmd file is knitted.

\begin{Shaded}
\begin{Highlighting}[]
\ControlFlowTok{if}\NormalTok{(}\OperatorTok{!}\KeywordTok{exists}\NormalTok{(}\StringTok{"storm.data"}\NormalTok{)) \{}
\NormalTok{    storm.data <-}\StringTok{ }\KeywordTok{read.csv}\NormalTok{(}\KeywordTok{bzfile}\NormalTok{(}\StringTok{"repdata_data_StormData.csv.bz2"}\NormalTok{),}\DataTypeTok{header =} \OtherTok{TRUE}\NormalTok{)}
\NormalTok{  \}}
\end{Highlighting}
\end{Shaded}

\hypertarget{examine-the-data-set}{%
\subsection{Examine the data set}\label{examine-the-data-set}}

\begin{Shaded}
\begin{Highlighting}[]
\KeywordTok{dim}\NormalTok{(storm.data)}
\end{Highlighting}
\end{Shaded}

\begin{verbatim}
## [1] 902297     37
\end{verbatim}

\begin{Shaded}
\begin{Highlighting}[]
\KeywordTok{str}\NormalTok{(storm.data)}
\end{Highlighting}
\end{Shaded}

\begin{verbatim}
## 'data.frame':    902297 obs. of  37 variables:
##  $ STATE__   : num  1 1 1 1 1 1 1 1 1 1 ...
##  $ BGN_DATE  : chr  "4/18/1950 0:00:00" "4/18/1950 0:00:00" "2/20/1951 0:00:00" "6/8/1951 0:00:00" ...
##  $ BGN_TIME  : chr  "0130" "0145" "1600" "0900" ...
##  $ TIME_ZONE : chr  "CST" "CST" "CST" "CST" ...
##  $ COUNTY    : num  97 3 57 89 43 77 9 123 125 57 ...
##  $ COUNTYNAME: chr  "MOBILE" "BALDWIN" "FAYETTE" "MADISON" ...
##  $ STATE     : chr  "AL" "AL" "AL" "AL" ...
##  $ EVTYPE    : chr  "TORNADO" "TORNADO" "TORNADO" "TORNADO" ...
##  $ BGN_RANGE : num  0 0 0 0 0 0 0 0 0 0 ...
##  $ BGN_AZI   : chr  "" "" "" "" ...
##  $ BGN_LOCATI: chr  "" "" "" "" ...
##  $ END_DATE  : chr  "" "" "" "" ...
##  $ END_TIME  : chr  "" "" "" "" ...
##  $ COUNTY_END: num  0 0 0 0 0 0 0 0 0 0 ...
##  $ COUNTYENDN: logi  NA NA NA NA NA NA ...
##  $ END_RANGE : num  0 0 0 0 0 0 0 0 0 0 ...
##  $ END_AZI   : chr  "" "" "" "" ...
##  $ END_LOCATI: chr  "" "" "" "" ...
##  $ LENGTH    : num  14 2 0.1 0 0 1.5 1.5 0 3.3 2.3 ...
##  $ WIDTH     : num  100 150 123 100 150 177 33 33 100 100 ...
##  $ F         : int  3 2 2 2 2 2 2 1 3 3 ...
##  $ MAG       : num  0 0 0 0 0 0 0 0 0 0 ...
##  $ FATALITIES: num  0 0 0 0 0 0 0 0 1 0 ...
##  $ INJURIES  : num  15 0 2 2 2 6 1 0 14 0 ...
##  $ PROPDMG   : num  25 2.5 25 2.5 2.5 2.5 2.5 2.5 25 25 ...
##  $ PROPDMGEXP: chr  "K" "K" "K" "K" ...
##  $ CROPDMG   : num  0 0 0 0 0 0 0 0 0 0 ...
##  $ CROPDMGEXP: chr  "" "" "" "" ...
##  $ WFO       : chr  "" "" "" "" ...
##  $ STATEOFFIC: chr  "" "" "" "" ...
##  $ ZONENAMES : chr  "" "" "" "" ...
##  $ LATITUDE  : num  3040 3042 3340 3458 3412 ...
##  $ LONGITUDE : num  8812 8755 8742 8626 8642 ...
##  $ LATITUDE_E: num  3051 0 0 0 0 ...
##  $ LONGITUDE_: num  8806 0 0 0 0 ...
##  $ REMARKS   : chr  "" "" "" "" ...
##  $ REFNUM    : num  1 2 3 4 5 6 7 8 9 10 ...
\end{verbatim}

\hypertarget{extracting-variables-of-interest-for-analysis-of-weather-impact-on-health-and-economy}{%
\subsection{Extracting variables of interest for analysis of weather
impact on health and
economy}\label{extracting-variables-of-interest-for-analysis-of-weather-impact-on-health-and-economy}}

From a list of variables in storm.data, these are columns of interest:

Health variables: * FATALITIES: approx. number of deaths * INJURIES:
approx. number of injuries

Economic variables:

PROPDMG: approx. property damags PROPDMGEXP: the units for property
damage value CROPDMG: approx. crop damages CROPDMGEXP: the units for
crop damage value Events - target variable:

EVTYPE: weather event (Tornados, Wind, Snow, Flood, etc..) Extract
variables of interest from original data set:

\begin{Shaded}
\begin{Highlighting}[]
\NormalTok{vars <-}\StringTok{ }\KeywordTok{c}\NormalTok{( }\StringTok{"EVTYPE"}\NormalTok{, }\StringTok{"FATALITIES"}\NormalTok{, }\StringTok{"INJURIES"}\NormalTok{, }\StringTok{"PROPDMG"}\NormalTok{, }\StringTok{"PROPDMGEXP"}\NormalTok{, }\StringTok{"CROPDMG"}\NormalTok{, }\StringTok{"CROPDMGEXP"}\NormalTok{)}
\NormalTok{mydata <-}\StringTok{ }\NormalTok{storm.data[, vars]}
\end{Highlighting}
\end{Shaded}

\begin{Shaded}
\begin{Highlighting}[]
\KeywordTok{tail}\NormalTok{(mydata)}
\end{Highlighting}
\end{Shaded}

\begin{verbatim}
##                EVTYPE FATALITIES INJURIES PROPDMG PROPDMGEXP CROPDMG CROPDMGEXP
## 902292 WINTER WEATHER          0        0       0          K       0          K
## 902293      HIGH WIND          0        0       0          K       0          K
## 902294      HIGH WIND          0        0       0          K       0          K
## 902295      HIGH WIND          0        0       0          K       0          K
## 902296       BLIZZARD          0        0       0          K       0          K
## 902297     HEAVY SNOW          0        0       0          K       0          K
\end{verbatim}

\hypertarget{checking-for-missing-values}{%
\subsection{Checking for missing
values}\label{checking-for-missing-values}}

Check for missing values in health variables - there is no NA's in the
data.

\begin{Shaded}
\begin{Highlighting}[]
\KeywordTok{sum}\NormalTok{(}\KeywordTok{is.na}\NormalTok{(mydata}\OperatorTok{$}\NormalTok{FATALITIES))}
\end{Highlighting}
\end{Shaded}

\begin{verbatim}
## [1] 0
\end{verbatim}

\begin{Shaded}
\begin{Highlighting}[]
\KeywordTok{sum}\NormalTok{(}\KeywordTok{is.na}\NormalTok{(mydata}\OperatorTok{$}\NormalTok{INJURIES))}
\end{Highlighting}
\end{Shaded}

\begin{verbatim}
## [1] 0
\end{verbatim}

Check for missing values in economic variables for ``size'' of damage -
there is no NA's in the data

\begin{Shaded}
\begin{Highlighting}[]
\KeywordTok{sum}\NormalTok{(}\KeywordTok{is.na}\NormalTok{(mydata}\OperatorTok{$}\NormalTok{PROPDMG))}
\end{Highlighting}
\end{Shaded}

\begin{verbatim}
## [1] 0
\end{verbatim}

\begin{Shaded}
\begin{Highlighting}[]
\KeywordTok{sum}\NormalTok{(}\KeywordTok{is.na}\NormalTok{(mydata}\OperatorTok{$}\NormalTok{CROPDMG))}
\end{Highlighting}
\end{Shaded}

\begin{verbatim}
## [1] 0
\end{verbatim}

Check for missing values in economic variables for units damage - there
is no NA's in the data.

\begin{Shaded}
\begin{Highlighting}[]
\KeywordTok{sum}\NormalTok{(}\KeywordTok{is.na}\NormalTok{(mydata}\OperatorTok{$}\NormalTok{PROPDMGEXP))}
\end{Highlighting}
\end{Shaded}

\begin{verbatim}
## [1] 0
\end{verbatim}

\begin{Shaded}
\begin{Highlighting}[]
\KeywordTok{sum}\NormalTok{(}\KeywordTok{is.na}\NormalTok{(mydata}\OperatorTok{$}\NormalTok{CROPDMGEXP))}
\end{Highlighting}
\end{Shaded}

\begin{verbatim}
## [1] 0
\end{verbatim}

\hypertarget{transforming-extracted-variables}{%
\subsection{Transforming extracted
variables}\label{transforming-extracted-variables}}

\begin{Shaded}
\begin{Highlighting}[]
\KeywordTok{sort}\NormalTok{(}\KeywordTok{table}\NormalTok{(mydata}\OperatorTok{$}\NormalTok{EVTYPE), }\DataTypeTok{decreasing =} \OtherTok{TRUE}\NormalTok{)[}\DecValTok{1}\OperatorTok{:}\DecValTok{10}\NormalTok{]}
\end{Highlighting}
\end{Shaded}

\begin{verbatim}
## 
##               HAIL          TSTM WIND  THUNDERSTORM WIND            TORNADO 
##             288661             219940              82563              60652 
##        FLASH FLOOD              FLOOD THUNDERSTORM WINDS          HIGH WIND 
##              54277              25326              20843              20212 
##          LIGHTNING         HEAVY SNOW 
##              15754              15708
\end{verbatim}

\begin{Shaded}
\begin{Highlighting}[]
\CommentTok{# create a new variable EVENT to transform variable EVTYPE in groups}
\NormalTok{mydata}\OperatorTok{$}\NormalTok{EVENT <-}\StringTok{ "OTHER"}
\CommentTok{# group by keyword in EVTYPE}
\NormalTok{mydata}\OperatorTok{$}\NormalTok{EVENT[}\KeywordTok{grep}\NormalTok{(}\StringTok{"HAIL"}\NormalTok{, mydata}\OperatorTok{$}\NormalTok{EVTYPE, }\DataTypeTok{ignore.case =} \OtherTok{TRUE}\NormalTok{)] <-}\StringTok{ "HAIL"}
\NormalTok{mydata}\OperatorTok{$}\NormalTok{EVENT[}\KeywordTok{grep}\NormalTok{(}\StringTok{"HEAT"}\NormalTok{, mydata}\OperatorTok{$}\NormalTok{EVTYPE, }\DataTypeTok{ignore.case =} \OtherTok{TRUE}\NormalTok{)] <-}\StringTok{ "HEAT"}
\NormalTok{mydata}\OperatorTok{$}\NormalTok{EVENT[}\KeywordTok{grep}\NormalTok{(}\StringTok{"FLOOD"}\NormalTok{, mydata}\OperatorTok{$}\NormalTok{EVTYPE, }\DataTypeTok{ignore.case =} \OtherTok{TRUE}\NormalTok{)] <-}\StringTok{ "FLOOD"}
\NormalTok{mydata}\OperatorTok{$}\NormalTok{EVENT[}\KeywordTok{grep}\NormalTok{(}\StringTok{"WIND"}\NormalTok{, mydata}\OperatorTok{$}\NormalTok{EVTYPE, }\DataTypeTok{ignore.case =} \OtherTok{TRUE}\NormalTok{)] <-}\StringTok{ "WIND"}
\NormalTok{mydata}\OperatorTok{$}\NormalTok{EVENT[}\KeywordTok{grep}\NormalTok{(}\StringTok{"STORM"}\NormalTok{, mydata}\OperatorTok{$}\NormalTok{EVTYPE, }\DataTypeTok{ignore.case =} \OtherTok{TRUE}\NormalTok{)] <-}\StringTok{ "STORM"}
\NormalTok{mydata}\OperatorTok{$}\NormalTok{EVENT[}\KeywordTok{grep}\NormalTok{(}\StringTok{"SNOW"}\NormalTok{, mydata}\OperatorTok{$}\NormalTok{EVTYPE, }\DataTypeTok{ignore.case =} \OtherTok{TRUE}\NormalTok{)] <-}\StringTok{ "SNOW"}
\NormalTok{mydata}\OperatorTok{$}\NormalTok{EVENT[}\KeywordTok{grep}\NormalTok{(}\StringTok{"TORNADO"}\NormalTok{, mydata}\OperatorTok{$}\NormalTok{EVTYPE, }\DataTypeTok{ignore.case =} \OtherTok{TRUE}\NormalTok{)] <-}\StringTok{ "TORNADO"}
\NormalTok{mydata}\OperatorTok{$}\NormalTok{EVENT[}\KeywordTok{grep}\NormalTok{(}\StringTok{"WINTER"}\NormalTok{, mydata}\OperatorTok{$}\NormalTok{EVTYPE, }\DataTypeTok{ignore.case =} \OtherTok{TRUE}\NormalTok{)] <-}\StringTok{ "WINTER"}
\NormalTok{mydata}\OperatorTok{$}\NormalTok{EVENT[}\KeywordTok{grep}\NormalTok{(}\StringTok{"RAIN"}\NormalTok{, mydata}\OperatorTok{$}\NormalTok{EVTYPE, }\DataTypeTok{ignore.case =} \OtherTok{TRUE}\NormalTok{)] <-}\StringTok{ "RAIN"}
\CommentTok{# listing the transformed event types }
\KeywordTok{sort}\NormalTok{(}\KeywordTok{table}\NormalTok{(mydata}\OperatorTok{$}\NormalTok{EVENT), }\DataTypeTok{decreasing =} \OtherTok{TRUE}\NormalTok{)}
\end{Highlighting}
\end{Shaded}

\begin{verbatim}
## 
##    HAIL    WIND   STORM   FLOOD TORNADO   OTHER  WINTER    SNOW    RAIN    HEAT 
##  289270  255362  113156   82686   60700   48970   19604   17660   12241    2648
\end{verbatim}

\begin{Shaded}
\begin{Highlighting}[]
\KeywordTok{sort}\NormalTok{(}\KeywordTok{table}\NormalTok{(mydata}\OperatorTok{$}\NormalTok{PROPDMGEXP), }\DataTypeTok{decreasing =} \OtherTok{TRUE}\NormalTok{)[}\DecValTok{1}\OperatorTok{:}\DecValTok{10}\NormalTok{]}
\end{Highlighting}
\end{Shaded}

\begin{verbatim}
## 
##             K      M      0      B      5      1      2      ?      m 
## 465934 424665  11330    216     40     28     25     13      8      7
\end{verbatim}

\begin{Shaded}
\begin{Highlighting}[]
\KeywordTok{sort}\NormalTok{(}\KeywordTok{table}\NormalTok{(mydata}\OperatorTok{$}\NormalTok{CROPDMGEXP), }\DataTypeTok{decreasing =} \OtherTok{TRUE}\NormalTok{)[}\DecValTok{1}\OperatorTok{:}\DecValTok{10}\NormalTok{]}
\end{Highlighting}
\end{Shaded}

\begin{verbatim}
## 
##             K      M      k      0      B      ?      2      m   <NA> 
## 618413 281832   1994     21     19      9      7      1      1
\end{verbatim}

\begin{Shaded}
\begin{Highlighting}[]
\NormalTok{mydata}\OperatorTok{$}\NormalTok{PROPDMGEXP <-}\StringTok{ }\KeywordTok{as.character}\NormalTok{(mydata}\OperatorTok{$}\NormalTok{PROPDMGEXP)}
\NormalTok{mydata}\OperatorTok{$}\NormalTok{PROPDMGEXP[}\KeywordTok{is.na}\NormalTok{(mydata}\OperatorTok{$}\NormalTok{PROPDMGEXP)] <-}\StringTok{ }\DecValTok{0} \CommentTok{# NA's considered as dollars}
\NormalTok{mydata}\OperatorTok{$}\NormalTok{PROPDMGEXP[}\OperatorTok{!}\KeywordTok{grepl}\NormalTok{(}\StringTok{"K|M|B"}\NormalTok{, mydata}\OperatorTok{$}\NormalTok{PROPDMGEXP, }\DataTypeTok{ignore.case =} \OtherTok{TRUE}\NormalTok{)] <-}\StringTok{ }\DecValTok{0} \CommentTok{# everything exept K,M,B is dollar}
\NormalTok{mydata}\OperatorTok{$}\NormalTok{PROPDMGEXP[}\KeywordTok{grep}\NormalTok{(}\StringTok{"K"}\NormalTok{, mydata}\OperatorTok{$}\NormalTok{PROPDMGEXP, }\DataTypeTok{ignore.case =} \OtherTok{TRUE}\NormalTok{)] <-}\StringTok{ "3"}
\NormalTok{mydata}\OperatorTok{$}\NormalTok{PROPDMGEXP[}\KeywordTok{grep}\NormalTok{(}\StringTok{"M"}\NormalTok{, mydata}\OperatorTok{$}\NormalTok{PROPDMGEXP, }\DataTypeTok{ignore.case =} \OtherTok{TRUE}\NormalTok{)] <-}\StringTok{ "6"}
\NormalTok{mydata}\OperatorTok{$}\NormalTok{PROPDMGEXP[}\KeywordTok{grep}\NormalTok{(}\StringTok{"B"}\NormalTok{, mydata}\OperatorTok{$}\NormalTok{PROPDMGEXP, }\DataTypeTok{ignore.case =} \OtherTok{TRUE}\NormalTok{)] <-}\StringTok{ "9"}
\NormalTok{mydata}\OperatorTok{$}\NormalTok{PROPDMGEXP <-}\StringTok{ }\KeywordTok{as.numeric}\NormalTok{(}\KeywordTok{as.character}\NormalTok{(mydata}\OperatorTok{$}\NormalTok{PROPDMGEXP))}
\NormalTok{mydata}\OperatorTok{$}\NormalTok{property.damage <-}\StringTok{ }\NormalTok{mydata}\OperatorTok{$}\NormalTok{PROPDMG }\OperatorTok{*}\StringTok{ }\DecValTok{10}\OperatorTok{^}\NormalTok{mydata}\OperatorTok{$}\NormalTok{PROPDMGEXP}

\NormalTok{mydata}\OperatorTok{$}\NormalTok{CROPDMGEXP <-}\StringTok{ }\KeywordTok{as.character}\NormalTok{(mydata}\OperatorTok{$}\NormalTok{CROPDMGEXP)}
\NormalTok{mydata}\OperatorTok{$}\NormalTok{CROPDMGEXP[}\KeywordTok{is.na}\NormalTok{(mydata}\OperatorTok{$}\NormalTok{CROPDMGEXP)] <-}\StringTok{ }\DecValTok{0} \CommentTok{# NA's considered as dollars}
\NormalTok{mydata}\OperatorTok{$}\NormalTok{CROPDMGEXP[}\OperatorTok{!}\KeywordTok{grepl}\NormalTok{(}\StringTok{"K|M|B"}\NormalTok{, mydata}\OperatorTok{$}\NormalTok{CROPDMGEXP, }\DataTypeTok{ignore.case =} \OtherTok{TRUE}\NormalTok{)] <-}\StringTok{ }\DecValTok{0} \CommentTok{# everything exept K,M,B is dollar}
\NormalTok{mydata}\OperatorTok{$}\NormalTok{CROPDMGEXP[}\KeywordTok{grep}\NormalTok{(}\StringTok{"K"}\NormalTok{, mydata}\OperatorTok{$}\NormalTok{CROPDMGEXP, }\DataTypeTok{ignore.case =} \OtherTok{TRUE}\NormalTok{)] <-}\StringTok{ "3"}
\NormalTok{mydata}\OperatorTok{$}\NormalTok{CROPDMGEXP[}\KeywordTok{grep}\NormalTok{(}\StringTok{"M"}\NormalTok{, mydata}\OperatorTok{$}\NormalTok{CROPDMGEXP, }\DataTypeTok{ignore.case =} \OtherTok{TRUE}\NormalTok{)] <-}\StringTok{ "6"}
\NormalTok{mydata}\OperatorTok{$}\NormalTok{CROPDMGEXP[}\KeywordTok{grep}\NormalTok{(}\StringTok{"B"}\NormalTok{, mydata}\OperatorTok{$}\NormalTok{CROPDMGEXP, }\DataTypeTok{ignore.case =} \OtherTok{TRUE}\NormalTok{)] <-}\StringTok{ "9"}
\NormalTok{mydata}\OperatorTok{$}\NormalTok{CROPDMGEXP <-}\StringTok{ }\KeywordTok{as.numeric}\NormalTok{(}\KeywordTok{as.character}\NormalTok{(mydata}\OperatorTok{$}\NormalTok{CROPDMGEXP))}
\NormalTok{mydata}\OperatorTok{$}\NormalTok{crop.damage <-}\StringTok{ }\NormalTok{mydata}\OperatorTok{$}\NormalTok{CROPDMG }\OperatorTok{*}\StringTok{ }\DecValTok{10}\OperatorTok{^}\NormalTok{mydata}\OperatorTok{$}\NormalTok{CROPDMGEXP}
\end{Highlighting}
\end{Shaded}

\begin{Shaded}
\begin{Highlighting}[]
\KeywordTok{sort}\NormalTok{(}\KeywordTok{table}\NormalTok{(mydata}\OperatorTok{$}\NormalTok{property.damage), }\DataTypeTok{decreasing =} \OtherTok{TRUE}\NormalTok{)[}\DecValTok{1}\OperatorTok{:}\DecValTok{10}\NormalTok{]}
\end{Highlighting}
\end{Shaded}

\begin{verbatim}
## 
##      0   5000  10000   1000   2000  25000  50000   3000  20000  15000 
## 663123  31731  21787  17544  17186  17104  13596  10364   9179   8617
\end{verbatim}

\begin{Shaded}
\begin{Highlighting}[]
\KeywordTok{sort}\NormalTok{(}\KeywordTok{table}\NormalTok{(mydata}\OperatorTok{$}\NormalTok{crop.damage), }\DataTypeTok{decreasing =} \OtherTok{TRUE}\NormalTok{)[}\DecValTok{1}\OperatorTok{:}\DecValTok{10}\NormalTok{]}
\end{Highlighting}
\end{Shaded}

\begin{verbatim}
## 
##      0   5000  10000  50000  1e+05   1000   2000  25000  20000  5e+05 
## 880198   4097   2349   1984   1233    956    951    830    758    721
\end{verbatim}

\hypertarget{analysis}{%
\subsection{Analysis}\label{analysis}}

\hypertarget{aggregating-events-for-public-health-variables}{%
\subsection{Aggregating events for public health
variables}\label{aggregating-events-for-public-health-variables}}

\begin{Shaded}
\begin{Highlighting}[]
\CommentTok{# aggregate FATALITIES and INJURIES by type of EVENT}
\NormalTok{agg.fatalites.and.injuries <-}\StringTok{ }\KeywordTok{ddply}\NormalTok{(mydata, .(EVENT), summarize, }\DataTypeTok{Total =} \KeywordTok{sum}\NormalTok{(FATALITIES }\OperatorTok{+}\StringTok{ }\NormalTok{INJURIES,  }\DataTypeTok{na.rm =} \OtherTok{TRUE}\NormalTok{))}
\NormalTok{agg.fatalites.and.injuries}\OperatorTok{$}\NormalTok{type <-}\StringTok{ "fatalities and injuries"}
  
\CommentTok{# aggregate FATALITIES by type of EVENT}
\NormalTok{agg.fatalities <-}\StringTok{ }\KeywordTok{ddply}\NormalTok{(mydata, .(EVENT), summarize, }\DataTypeTok{Total =} \KeywordTok{sum}\NormalTok{(FATALITIES, }\DataTypeTok{na.rm =} \OtherTok{TRUE}\NormalTok{))}
\NormalTok{agg.fatalities}\OperatorTok{$}\NormalTok{type <-}\StringTok{ "fatalities"}

\CommentTok{# aggregate INJURIES by type of EVENT}
\NormalTok{agg.injuries <-}\StringTok{ }\KeywordTok{ddply}\NormalTok{(mydata, .(EVENT), summarize, }\DataTypeTok{Total =} \KeywordTok{sum}\NormalTok{(INJURIES, }\DataTypeTok{na.rm =} \OtherTok{TRUE}\NormalTok{))}
\NormalTok{agg.injuries}\OperatorTok{$}\NormalTok{type <-}\StringTok{ "injuries"}

\CommentTok{# combine all}
\NormalTok{agg.health <-}\StringTok{ }\KeywordTok{rbind}\NormalTok{(agg.fatalities, agg.injuries)}

\NormalTok{health.by.event <-}\StringTok{ }\KeywordTok{join}\NormalTok{ (agg.fatalities, agg.injuries, }\DataTypeTok{by=}\StringTok{"EVENT"}\NormalTok{, }\DataTypeTok{type=}\StringTok{"inner"}\NormalTok{)}
\NormalTok{health.by.event}
\end{Highlighting}
\end{Shaded}

\begin{verbatim}
##      EVENT Total       type Total     type
## 1    FLOOD  1524 fatalities  8602 injuries
## 2     HAIL    15 fatalities  1371 injuries
## 3     HEAT  3138 fatalities  9224 injuries
## 4    OTHER  2626 fatalities 12224 injuries
## 5     RAIN   114 fatalities   305 injuries
## 6     SNOW   164 fatalities  1164 injuries
## 7    STORM   416 fatalities  5339 injuries
## 8  TORNADO  5661 fatalities 91407 injuries
## 9     WIND  1209 fatalities  9001 injuries
## 10  WINTER   278 fatalities  1891 injuries
\end{verbatim}

\hypertarget{aggregating-events-for-economic-variables}{%
\subsection{Aggregating events for economic
variables}\label{aggregating-events-for-economic-variables}}

\begin{Shaded}
\begin{Highlighting}[]
\CommentTok{# aggregate PropDamage and CropDamage by type of EVENT}
\NormalTok{agg.propdmg.and.cropdmg <-}\StringTok{ }\KeywordTok{ddply}\NormalTok{(mydata, .(EVENT), summarize, }\DataTypeTok{Total =} \KeywordTok{sum}\NormalTok{(property.damage }\OperatorTok{+}\StringTok{ }\NormalTok{crop.damage,  }\DataTypeTok{na.rm =} \OtherTok{TRUE}\NormalTok{))}
\NormalTok{agg.propdmg.and.cropdmg}\OperatorTok{$}\NormalTok{type <-}\StringTok{ "property and crop damage"}

\CommentTok{# aggregate PropDamage by type of EVENT}
\NormalTok{agg.prop <-}\StringTok{ }\KeywordTok{ddply}\NormalTok{(mydata, .(EVENT), summarize, }\DataTypeTok{Total =} \KeywordTok{sum}\NormalTok{(property.damage, }\DataTypeTok{na.rm =} \OtherTok{TRUE}\NormalTok{))}
\NormalTok{agg.prop}\OperatorTok{$}\NormalTok{type <-}\StringTok{ "property"}

\CommentTok{# aggregate INJURIES by type of EVENT}
\NormalTok{agg.crop <-}\StringTok{ }\KeywordTok{ddply}\NormalTok{(mydata, .(EVENT), summarize, }\DataTypeTok{Total =} \KeywordTok{sum}\NormalTok{(crop.damage, }\DataTypeTok{na.rm =} \OtherTok{TRUE}\NormalTok{))}
\NormalTok{agg.crop}\OperatorTok{$}\NormalTok{type <-}\StringTok{ "crop"}

\CommentTok{# combine all}
\NormalTok{agg.economic <-}\StringTok{ }\KeywordTok{rbind}\NormalTok{(agg.prop, agg.crop)}


\NormalTok{economic.by.event <-}\StringTok{ }\KeywordTok{join}\NormalTok{ (agg.prop, agg.crop, }\DataTypeTok{by=}\StringTok{"EVENT"}\NormalTok{, }\DataTypeTok{type=}\StringTok{"inner"}\NormalTok{)}
\NormalTok{economic.by.event}
\end{Highlighting}
\end{Shaded}

\begin{verbatim}
##      EVENT        Total     type       Total type
## 1    FLOOD 167502193929 property 12266906100 crop
## 2     HAIL  15733043048 property  3046837473 crop
## 3     HEAT     20325750 property   904469280 crop
## 4    OTHER  97246712337 property 23588880870 crop
## 5     RAIN   3270230192 property   919315800 crop
## 6     SNOW   1024169752 property   134683100 crop
## 7    STORM  66304415393 property  6374474888 crop
## 8  TORNADO  58593098029 property   417461520 crop
## 9     WIND  10847166618 property  1403719150 crop
## 10  WINTER   6777295251 property    47444000 crop
\end{verbatim}

\hypertarget{results}{%
\subsection{Results}\label{results}}

\hypertarget{across-the-united-states-which-types-of-events-are-most-harmful-with-respect-to-population-health}{%
\subsection{Across the United States, which types of events are most
harmful with respect to population
health?}\label{across-the-united-states-which-types-of-events-are-most-harmful-with-respect-to-population-health}}

\begin{Shaded}
\begin{Highlighting}[]
\CommentTok{# transform EVENT to factor variable for health variables}
\NormalTok{agg.health}\OperatorTok{$}\NormalTok{EVENT <-}\StringTok{ }\KeywordTok{as.factor}\NormalTok{(agg.health}\OperatorTok{$}\NormalTok{EVENT)}

\CommentTok{# plot FATALITIES and INJURIES by EVENT}
\NormalTok{health.plot <-}\StringTok{ }\KeywordTok{ggplot}\NormalTok{(agg.health, }\KeywordTok{aes}\NormalTok{(}\DataTypeTok{x =}\NormalTok{ EVENT, }\DataTypeTok{y =}\NormalTok{ Total, }\DataTypeTok{fill =}\NormalTok{ type)) }\OperatorTok{+}\StringTok{ }\KeywordTok{geom_bar}\NormalTok{(}\DataTypeTok{stat =} \StringTok{"identity"}\NormalTok{) }\OperatorTok{+}
\StringTok{  }\KeywordTok{coord_flip}\NormalTok{() }\OperatorTok{+}
\StringTok{  }\KeywordTok{xlab}\NormalTok{(}\StringTok{"Event Type"}\NormalTok{) }\OperatorTok{+}\StringTok{ }
\StringTok{  }\KeywordTok{ylab}\NormalTok{(}\StringTok{"Total number of health impact"}\NormalTok{) }\OperatorTok{+}
\StringTok{  }\KeywordTok{ggtitle}\NormalTok{(}\StringTok{"Weather event types impact on public health"}\NormalTok{) }\OperatorTok{+}
\StringTok{  }\KeywordTok{theme}\NormalTok{(}\DataTypeTok{plot.title =} \KeywordTok{element_text}\NormalTok{(}\DataTypeTok{hjust =} \FloatTok{0.5}\NormalTok{))}
\KeywordTok{print}\NormalTok{(health.plot)  }
\end{Highlighting}
\end{Shaded}

\includegraphics{Reproducible-Research---Course-Project-2_files/figure-latex/unnamed-chunk-21-1.pdf}

The most harmful weather event for health (in number of total fatalites
and injuries) is, by far, a tornado.

\hypertarget{across-the-united-states-which-types-of-events-have-the-greatest-economic-consequences}{%
\subsection{Across the United States, which types of events have the
greatest economic
consequences?}\label{across-the-united-states-which-types-of-events-have-the-greatest-economic-consequences}}

\begin{Shaded}
\begin{Highlighting}[]
\CommentTok{# # transform EVENT to factor variable for economic variables}
\NormalTok{agg.economic}\OperatorTok{$}\NormalTok{EVENT <-}\StringTok{ }\KeywordTok{as.factor}\NormalTok{(agg.economic}\OperatorTok{$}\NormalTok{EVENT)}

\CommentTok{# plot PROPERTY damage and CROP damage by EVENT}
\NormalTok{economic.plot <-}\StringTok{ }\KeywordTok{ggplot}\NormalTok{(agg.economic, }\KeywordTok{aes}\NormalTok{(}\DataTypeTok{x =}\NormalTok{ EVENT, }\DataTypeTok{y =}\NormalTok{ Total, }\DataTypeTok{fill =}\NormalTok{ type)) }\OperatorTok{+}\StringTok{ }\KeywordTok{geom_bar}\NormalTok{(}\DataTypeTok{stat =} \StringTok{"identity"}\NormalTok{) }\OperatorTok{+}
\StringTok{  }\KeywordTok{coord_flip}\NormalTok{() }\OperatorTok{+}
\StringTok{  }\KeywordTok{xlab}\NormalTok{(}\StringTok{"Event Type"}\NormalTok{) }\OperatorTok{+}\StringTok{ }
\StringTok{  }\KeywordTok{ylab}\NormalTok{(}\StringTok{"Total damage in dollars"}\NormalTok{) }\OperatorTok{+}
\StringTok{  }\KeywordTok{ggtitle}\NormalTok{(}\StringTok{"Weather event types impact on property and crop damage"}\NormalTok{) }\OperatorTok{+}
\StringTok{  }\KeywordTok{theme}\NormalTok{(}\DataTypeTok{plot.title =} \KeywordTok{element_text}\NormalTok{(}\DataTypeTok{hjust =} \FloatTok{0.5}\NormalTok{))}
\KeywordTok{print}\NormalTok{(economic.plot) }
\end{Highlighting}
\end{Shaded}

\includegraphics{Reproducible-Research---Course-Project-2_files/figure-latex/unnamed-chunk-22-1.pdf}

The most devastating weather event with the greatest economic
cosequences (to property and crops) is a flood.

\end{document}
